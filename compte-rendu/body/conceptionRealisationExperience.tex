
\section{Conception et réalisation de notre expérience}
% Indiquez ici le ou les objectifs de votre propre expérience (voir la partie 4 du sujet de TP, éventuellement enrichie de vos propres questions additionnelles).
Nous souhaitons tester l'influence de $n$, $m$ et $k$ sur les valeurs de $count1$, $count2$ et $count3$.

\subsection{Plan d'expérience}
% Décrivez ici votre plan d'expérience :
% \begin{itemize}
% 	\item Variables mesurées
% 	\item Paramètres que vous avez fait varier, et les valeurs choisies (justifiez). On peut considérer ici que les fichiers d'entrée font partie des paramètres, donc inpliquez lesquels vous avez choisis et expliquez pourquoi.
% 	\item Mode de combinaison des valeurs de paramètres (par exemple : full factorial design, si toutes les combinaisons possibles ont été testées)
% 	\item Nombre de runs effectués pour chaque combinaison de valeurs de paramètre
% 	\item Environnement de test
% \end{itemize}

% Si vous avez utilisé un outil de type ``cahier de laboratoire'' (papier ou numérique), indiquez-le ici, en incluant une figure (scan ou copie d'écran).

Pour réaliser les expériences et éviter le problème présenté dans l'analyse des résultats de l'expérience fournie vis à vis de l'initialisation de l'algorithme de génération de nombres aléatoires, nous avons utilisé une fonction de la librairie $windows.h$ qui retournera une valeur différente à chaque appel.

\begin{lstlisting}[language=C]
#include <windows.h>

/* [...] */

unsigned __int64 rndSeed;
QueryPerformanceCounter((LARGE_INTEGER *)&rndSeed);
srand(rndSeed);

/* [...] */
\end{lstlisting}


\subsubsection{Variables mesurées}

Les variables mesurées sont les suivantes :

\begin{table}[h!]
	\centering
	\caption{Variables mesurées}
	\label{tab:variablesMesureesExperience}
	\begin{tabular}{c|c}
		\toprule
		Variable & Description\\
		\midrule
		count1 & Nombre d'appels à la méthode $wordncmp$\\
		count2 & Nombre d'appels à la méthode $wordncmp$\\
		count3 & Nombre d'appels à la méthode $wordncmp$\\
		usr & Temps approximatif utilisé par le programme en lui même\\
		sys & Temps approximatif utilisé par le système d'exploitation\\
		time & Temps approximatif entre le début et la fin de l'exécution du programme\\
		\bottomrule
	\end{tabular}
\end{table}

\subsubsection{Paramètres}
Il a été réalisé 10 tests par jeu de paramètre.

\begin{table}[h!]
	\centering
	\caption{Paramètres numériques}
	\label{tab:parametresNumeriquesRealisee}
	\begin{tabular}{c|ccc}
		\toprule
		Paramètre & Valeur min & Valeur max & Incrément\\
		\midrule
		m & 100 & 1000000 & $*10$\\
		k & 2 & 7 & $+1$\\
		\bottomrule
	\end{tabular}
\end{table}

\begin{table}[h!]
	\centering
	\caption{Valeurs de $n$ choisies.}
	\label{tab:valeursDeNChoisiesRealisee}
	\begin{tabular}{c|ccc}
		\toprule
		Fichier & Nombre de mots\\
		\midrule
		Le catéchumène & 6640\\
		Zig ou la destinée & 26046\\
		Venus Boy & 39213\\
		Battlefields of the Marne & 75166\\
		Madame Bovary & 112451\\
		Don Quixote & 404461\\
		Histoire des salons de Paris & 492732\\
		\bottomrule
	\end{tabular}
\end{table}


\subsubsection{Mode de combinaison des valeurs}
Les tests ont été effectués à partir de plages de paramètres. Toutes les valeurs possibles n'ont pas été utilisées, étant donné que certaines valeurs (par exemple : de grandes valeurs de $k$) ne sont plus adaptées à une utilisation cohérente du programme. C'est à dire que l'utilisation de tels paramètres donnerait un résultat d'exploitation (et non en terme de temps, etc...) non désiré dans une utilisation normale du programme.

\subsubsection{Nombre de runs}
Nous avons réalisé 2100 runs.

\subsubsection{Environnement de test}
\begin{table}[h!]
	\centering
	\caption{Informations sur la machine utilisée}
	\label{tab:environnementExperience}
	\begin{tabular}{c|c}
		\toprule
		OS & Microsoft Windows 10 Professionnel 10.0.10240\\
		Processeur & Intel(R) Core(TM) i7-4702MQ CPU @ 2.20GHz 4 coeurs 8 processeurs logiques\\
		Memoire & 7.66GB DDR3\\
		Disque & 1TB HDD Toshiba MQ01ABD100\\
		Compilateur & gcc 4.8.1 (Windows 10)\\
		\bottomrule
	\end{tabular}
\end{table}


\subsection{Analyse des résultats}
% Synthétisez les résultats obtenus à l'aide de trois figures maximum, et commentez-les. Les relations obtenues correspondent-elles à ce qu'on attend étant donné l'algorithme ? Quelle est l'influence de chaque paramètre ? Obtenez-vous les mêmes résultats qu'avec l'expérience fournie ?


Count1 $\sim$ n + k + m
\begin{verbatim}
Estimate Std. Error t value Pr(>|t|)    
(Intercept) -6.856e+05  2.629e+05  -2.608  0.00917 ** 
n            1.209e+01  4.757e-01  25.421  < 2e-16 ***
k            7.583e+04  5.109e+04   1.484  0.13789    
m            1.718e-02  2.233e-03   7.694 2.18e-14 ***
\end{verbatim}
Nous pouvons voir que seul n est fortement lié à Count1.\\
$R^2$ de $count1$ $\sim$ $n$ : 0.2301\\

Count2 $\sim$ n + k + m
\begin{verbatim}
Estimate Std. Error t value Pr(>|t|)    
(Intercept) -2.753e+05  7.855e+03  -35.05   <2e-16 ***
n            2.247e+01  1.422e-02 1580.88   <2e-16 ***
k            1.853e+04  1.527e+03   12.14   <2e-16 ***
m            4.283e-17  6.673e-05    0.00        1    
\end{verbatim}
Nous pouvons voir que seul n et k sont fortement liés à Count2.\\
$R^2$ de $count2$ $\sim$ $n$ : 0.9991\\
$R^2$ de $count2$ $\sim$ $k$ : -0.0004177\\

Count3 $\sim$ n + k + m
\begin{verbatim}
Estimate Std. Error t value Pr(>|t|)    
(Intercept)  3.356e+06  3.457e+05   9.708  < 2e-16 ***
n            8.389e+00  6.256e-01  13.409  < 2e-16 ***
k           -8.403e+05  6.719e+04 -12.506  < 2e-16 ***
m            1.241e-02  2.937e-03   4.225 2.49e-05 ***
\end{verbatim}
Nous pouvons voir que n et k sont étroitement liés à Count3.\\
$R^2$ de $count3$ $\sim$ $n$ : 0.07295\\
$R^2$ de $count3$ $\sim$ $k$ : 0.06339\\

Coefficients de corrélation :
\begin{verbatim}
count1        count2      count3
n 0.48010732  9.995515e-01  0.27090054
k 0.02803182  7.675098e-03 -0.25266565
m 0.14530334 -2.633931e-20  0.08536216
\end{verbatim}


Nous n'obtenons pas les mêmes résultats que l'expérience fournie. Cela peut venir de la stochasticité du programme, du problème d'initialisation de l'algorithme aléatoire pour l'expérience fournie et/ou des fichiers utilisés qui peuvent avoir des caractéristiques particulières.
