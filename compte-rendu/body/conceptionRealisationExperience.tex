
\section{Conception et réalisation de votre expérience}
% Indiquez ici le ou les objectifs de votre propre expérience (voir la partie 4 du sujet de TP, éventuellement enrichie de vos propres questions additionnelles).

\subsection{Plan d'expérience}
% Décrivez ici votre plan d'expérience :
% \begin{itemize}
% 	\item Variables mesurées
% 	\item Paramètres que vous avez fait varier, et les valeurs choisies (justifiez). On peut considérer ici que les fichiers d'entrée font partie des paramètres, donc inpliquez lesquels vous avez choisis et expliquez pourquoi.
% 	\item Mode de combinaison des valeurs de paramètres (par exemple : full factorial design, si toutes les combinaisons possibles ont été testées)
% 	\item Nombre de runs effectués pour chaque combinaison de valeurs de paramètre
% 	\item Environnement de test
% \end{itemize}

% Si vous avez utilisé un outil de type ``cahier de laboratoire'' (papier ou numérique), indiquez-le ici, en incluant une figure (scan ou copie d'écran).

\subsection{Analyse des résultats}
% Synthétisez les résultats obtenus à l'aide de trois figures maximum, et commentez-les. Les relations obtenues correspondent-elles à ce qu'on attend étant donné l'algorithme ? Quelle est l'influence de chaque paramètre ? Obtenez-vous les mêmes résultats qu'avec l'expérience fournie ?
