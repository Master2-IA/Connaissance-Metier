
\section{Présentation sommaire du programme à évaluer}

\subsection{Données à fournir en entrée}

On fournit, en entrée, un fichier texte venant d'un roman.
Ce fichier contient uniquement le texte du roman. Il ne doit pas avoir de numéro de page ou d'éléments qui ne sont pas liés au texte lui-même.

\subsection{Données fournies en sortie}

En sortie, nous obtenons un texte composé de $m$ mots (si possible) issus du texte d'entrée.
Le résultat est une texte qui donne l'illusion d'avoir été rédigé par une personne.
Cette illusion va être plus ou moins efficace en fonction du paramètre $k$.
Une valeur trop grande du paramètre $k$ va donner un texte identique au texte d'origine.

\subsection{Algorithme}

% Décrivez en français ce que vous avez compris de l'algorithme. Expliquez entre autres ce que représentent $n$, $m$, et $k$, et en quoi l'algorithme est stochastique. Vous pouvez en plus écrire l'algorithme sous forme de pseudocode (bonus). Voir  \url{https://en.wikibooks.org/wiki/LaTeX/Algorithms#Typesetting_using_the_algorithmic_package} pour plus d'informations.

L'algorithme découpe le texte d'entrée en mots et les trie.
L'algorithme va successivement trouver l'index du mot anciennement utilisé, jusqu'à ce qu'il n'y ait plus de mots disponibles ou jusqu'à ce qu'il ait atteint le nombre de mots de sortie désiré. Ensuite, il sélectionne un nouveau mot à partir de cet index, et pour terminer, il écrit ce mot dans la sortie désirée.



\begin{itemize}
	\item $n$ représente le nombre de mots mesurés dans le fichier d'entrée.
	\item $m$ représente le nombre de mots que l'on souhaite avoir en sortie.
	\item $k$ représente le nombre de mots par phrase.
\end{itemize}

L'algorithme est stochastique dans sa sélection des mots. Néanmoins, sa stochasticité peut être annulée par l'utilisation d'une mauvaise graine pour l'initialisation de la fonction d'aléatoire (srand).
