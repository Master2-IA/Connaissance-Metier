
\section{Présentation sommaire du programme à évaluer}

\subsection{Données à fournir en entrée}

\subsection{Données fournies en sortie}

\subsection{Algorithme}

% Décrivez en français ce que vous avez compris de l'algorithme. Expliquez entre autres ce que représentent $n$, $m$, et $k$, et en quoi l'algorithme est stochastique. Vous pouvez en plus écrire l'algorithme sous forme de pseudocode (bonus). Voir  \url{https://en.wikibooks.org/wiki/LaTeX/Algorithms#Typesetting_using_the_algorithmic_package} pour plus d'informations.


\begin{itemize}
	\item $n$ représente le nombre de mots mesurés dans le fichier d'entrée.
	\item $m$ représente le nombre de mots que l'on souhaite avoir en sortie.
	\item $k$ représente le nombre de mots à passer $+ 1$ dans le texte d'entrée.
\end{itemize}



\begin{algorithm}
	\caption{<your caption for this algorithm>}
	\label{<your label for references later in your document>}
	\begin{algorithmic}
		% A remplacer par votre pseudocode
		\If {$i\geq maxval$}
		\State $i\gets 0$
		\Else
		\If {$i+k\leq maxval$}
		\State $i\gets i+k$
		\EndIf
		\EndIf
	\end{algorithmic}
\end{algorithm}
