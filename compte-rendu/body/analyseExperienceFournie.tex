
\section{Analyse de l'expérience fournie par R. Thion}

% Rappelez le ou les objectifs de cette expérience (voir la partie 4 du sujet de TP) : à quelle(s) question(s) va-t-on répondre ?
Nous souhaitons tester l'influence de $n$, $m$ et $k$ sur les valeurs de $count1$, $count2$ et $count3$.

\subsection{Plan d'expérience}
% Rappelez ici le plan de l'expérience qui vous a été fournie :
% \begin{itemize}
% 	\item Variables mesurées
% 	\item Paramètres que l'on a fait varier, et les valeurs choisies (sans oublier les fichiers d'entrée).
% 	\item Mode de combinaison des valeurs de paramètres (par exemple : full factorial design, si toutes les combinaisons possibles ont été testées)
% 	\item Nombre de runs effectués pour chaque combinaison de valeurs de paramètre
% 	\item Environnement de test
% \end{itemize}

\subsubsection{Variables mesurées}

Les variables mesurées sont les suivantes :

\begin{table}[h!]
	\centering
	\caption{Variables mesurées.}
	\label{tab:variablesMesureesFournie}
	\begin{tabular}{c|c}
		\toprule
		Variable & Description\\
		\midrule
		count1 & Nombre d'appels à la méthode $wordncmp$.\\
		count2 & Nombre d'appels à la méthode $wordncmp$.\\
		count3 & Nombre d'appels à la méthode $wordncmp$.\\
		usr & Temps approximatif utilisé par le programme en lui même.\\
		sys & Temps approximatif utilisé par le système d'exploitation.\\
		time & Temps approximatif entre le début et la fin de l'exécution du programme.\\
		\bottomrule
	\end{tabular}
\end{table}

\subsubsection{Paramètres}
Il a été réalisé 20 tests par jeu de paramètre.

\begin{table}[h!]
	\centering
	\caption{Paramètres numériques.}
	\label{tab:parametresNumeriquesFournie}
	\begin{tabular}{c|ccc}
		\toprule
		Paramètre & Valeur min & Valeur max & Incrément\\
		\midrule
		m & 100 & 1000000 & $*10$\\
		k & 2 & 7 & $+1$\\
		\bottomrule
	\end{tabular}
\end{table}

\begin{table}[h!]
	\centering
	\caption{Valeurs de $n$ choisies.}
	\label{tab:valeursDeNChoisiesFournie}
	\begin{tabular}{c|ccc}
		\toprule
		Fichier & Nombre de mots\\
		\midrule
		book1\_assommoir.txt & 91033\\
		book2\_sherlock.txt & 104410\\
		book3\_wonderland.txt & 26438\\
		book4\_don-quixote\_400k.txt & 404461\\
		book4\_don-quixote\_40k.txt & 40171\\
		book4\_don-quixote\_4k.txt & 4030\\
		book5\_allbooks.txt & 626342\\
		\bottomrule
	\end{tabular}
\end{table}


\subsubsection{Mode de combinaison des valeurs}
Les tests ont été effectués à partir de plages de paramètres. Toutes les valeurs possibles n'ont pas été utilisées étant donné que certaines valeurs (par exemple : de grandes valeurs de $k$) ne sont plus adaptées à une utilisation cohérente du programme. C'est à dire que l'utilisation de tels paramètres donnerait un résultat d'exploitation (et non en terme de temps, etc...) indésiré dans une utilisation normale du programme.

\subsubsection{Nombre de runs}
4200

\subsubsection{Environnement de test}
\begin{table}[h!]
	\centering
	\caption{Informations sur la machine utilisée.}
	\label{tab:environnementFournie}
	\begin{tabular}{c|c}
		\toprule
		OS & 4.2.0-23-generic \#28-Ubuntu SMP x86\_64 GNU/Linux (Ubuntu 15.10)\\
		Processeur & Intel(R) Core(TM) i7-5600U CPU @ 2.60GHz (dual-core)\\
		Memoire & 8GB 1600MHz DDR3L\\
		Disque & 180GB M2 SATA-3 Solid State Drive\\
		Compilateur & g++ (Ubuntu 5.2.1-22ubuntu2) 5.2.1 20151010\\
		\bottomrule
	\end{tabular}
\end{table}

\subsection{Analyse des résultats}
% Synthétisez les résultats obtenus à l'aide de trois figures maximum. Commentez-les et indiquez dans le texte principal les coefficients de détermination des régressions linéaires ($R^2$), les coefficients de corrélation et leurs p-values, etc.  Concluez sur les questions posées au début de cette section.

Count1 ~ n + k + m
\begin{verbatim}
              Estimate Std. Error  t value Pr(>|t|)    
              (Intercept) -1.756e+05  4.624e+03  -37.980   <2e-16 ***
              n            1.811e+01  7.032e-03 2575.692   <2e-16 ***
              k            3.159e+00  9.036e+02    0.003    0.997    
              m            9.230e-16  3.950e-03    0.000    1.000   
\end{verbatim}
Nous pouvons voir que seul n est fortement lié à count1.
$R^2$ de $count1$ ~ $n$ : 0.9994

Count2 ~ n + k + m
\begin{verbatim}
              Estimate Std. Error t value Pr(>|t|)    
              (Intercept) -3.487e+05  8.466e+04  -4.119 3.88e-05 ***
              n            2.988e+00  1.287e-01  23.207  < 2e-16 ***
              k           -2.293e+03  1.654e+04  -0.139     0.89    
              m            2.206e+00  7.231e-02  30.505  < 2e-16 ***
\end{verbatim}
Nous pouvons voir que seul n et m sont fortement liés à count2.
$R^2$ de $count2$ ~ $n$ : 0.09485
$R^2$ de $count2$ ~ $m$ : 0.1641

Count3 ~ n + k + m
\begin{verbatim}
              Estimate Std. Error t value Pr(>|t|)    
              (Intercept)  1.134e+06  2.007e+05   5.652 1.69e-08 ***
              n            3.291e+00  3.051e-01  10.785  < 2e-16 ***
              k           -3.658e+05  3.921e+04  -9.328  < 2e-16 ***
              m            1.848e+00  1.714e-01  10.780  < 2e-16 ***
\end{verbatim}
Nous pouvons voir que n, k et m sont étroitement liés à count3.
$R^2$ de $count3$ ~ $n$ : 0.02553 
$R^2$ de $count3$ ~ $k$ : 0.01904 
$R^2$ de $count3$ ~ $m$ : 0.0255


Coefficients de corrélation :
\begin{verbatim}
count1       count2     count3
n 9.996839e-01  0.308325932  0.1605025
k 1.356917e-06 -0.001841625 -0.1388157
m 1.115266e-19  0.405293749  0.1604183
\end{verbatim}


En regardant les résultats obtenus sur les jeux de paramètres identiques, on peut constater que les résultats obtenus sont les mêmes. Cela peut s'expliquer si les tests ont été réalisés avec une graine aléatoire ($srand(...)$) en seconde telle que $time(NULL)$. En effet, cette dernière fonction retourne le temps en seconde, ce qui fourni à $srand$ la même valeur tant que les tests sont réalisés la même seconde. Étant donné que l'exécution du programme a nécessité peu de temps et que ceux-ci ont été enchainés en moins d'une seconde, cela explique ce résultat. Par conséquent, il est préférable d'utiliser une graine qui dépend du nombre de cycles du processeur et non du temps lorsque l'on effectue de nombreux tests à la suite. Nous pouvons donc conclure qu'environ 210 résultats sur 4200 sont réellement intéressant, les autres étant des doublons.
