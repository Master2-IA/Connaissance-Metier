
\section{Opération dominante}

\subsection{Paramètres utilisés dans toute cette section}

\subsection{Identification de l'opération dominante}
% Indiquez le ou les outils de profilage utilisés pour trouver l'opération dominante. Pour appuyer votre réponse, vous pouvez copier-coller un extrait du rapport de profilage, ou inclure une capture d'écran comme figure.

Pour déterminer l'opération dominante au sein du programme, nous avons utilisé gprof.

% http://sourceware.org/binutils/docs/gprof/Implementation.html

\begin{lstlisting}[language=bash]
 $ gcc -pg -o markovPG markov.c
 $ ./markovPG.exe 3 10000 < ./don-quixote.txt > outPG.txt
 $ gprof markovPG.exe gmon.out > pg2.out
\end{lstlisting}

\begin{verbatim}
%   cumulative   self              self     total           
time   seconds   seconds    calls  ns/call  ns/call  name    
53.57      0.15     0.15  7746432    19.36    19.36  wordncmp
17.86      0.20     0.05                             sortcmp
14.29      0.24     0.04                             __fentry__
14.29      0.28     0.04                             _mcount_private
0.00      0.28     0.00    30000     0.00     0.00  skip
0.00      0.28     0.00    10000     0.00     0.00  writeword
\end{verbatim}
Nous pouvons voir que la fonction $wordncmp$ est la plus appellée avec 7746432 appels.





\subsection{Comptage des appels à l'opération dominante}
% Indiquez ici les lignes de code que vous avez modifiées pour compter les appels à l'opération dominante.

Nous avons ajouté, de façon globale, un tableau de 3 cases de long int.
Nous incrémentons ensuite à chaque appel la case du tableau correspondante.


\begin{lstlisting}[language=C]
/* [...] */
up = nword;
while(lo+1 != up)
{
	mid = (lo + up) / 2;
	nbCall[0]++;
	if(wordncmp(word[mid], phrase) < 0)
		lo = mid;
	else
		up = mid;
}
/* [...] */
\end{lstlisting}

\begin{lstlisting}[language=C]
/* [...] */
nbCall[1]++;
for(i = 0; wordncmp(phrase, word[up+i]) == 0; i++)
{
	if(rand() % (i+1) == 0)
		p = word[up+i];
	nbCall[1]++;
}
/* [...] */
\end{lstlisting}
Il ne faut pas oublier de mettre une incrémentation avant ou après la boucle for car celle-ci va faire $x$ tours de boucle, impliquant d'avoir $x$ fois la condition qui vaut $vrai$ et $1$ fois qui vaut $faux$. Nous avons donc $x + 1$ appels à $wordncmp$.


\begin{lstlisting}[language=C]
/* [...] */
/* called by system qsort */
int sortcmp(const void* p, const void* q)
{
	char** p1 = (char**)p;
	char** q1 = (char**)q;
	nbCall[2]++;
	return wordncmp(*p1, *q1);
}
/* [...] */
\end{lstlisting}

